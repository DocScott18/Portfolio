% Options for packages loaded elsewhere
\PassOptionsToPackage{unicode}{hyperref}
\PassOptionsToPackage{hyphens}{url}
%
\documentclass[
]{article}
\usepackage{amsmath,amssymb}
\usepackage{iftex}
\ifPDFTeX
  \usepackage[T1]{fontenc}
  \usepackage[utf8]{inputenc}
  \usepackage{textcomp} % provide euro and other symbols
\else % if luatex or xetex
  \usepackage{unicode-math} % this also loads fontspec
  \defaultfontfeatures{Scale=MatchLowercase}
  \defaultfontfeatures[\rmfamily]{Ligatures=TeX,Scale=1}
\fi
\usepackage{lmodern}
\ifPDFTeX\else
  % xetex/luatex font selection
\fi
% Use upquote if available, for straight quotes in verbatim environments
\IfFileExists{upquote.sty}{\usepackage{upquote}}{}
\IfFileExists{microtype.sty}{% use microtype if available
  \usepackage[]{microtype}
  \UseMicrotypeSet[protrusion]{basicmath} % disable protrusion for tt fonts
}{}
\makeatletter
\@ifundefined{KOMAClassName}{% if non-KOMA class
  \IfFileExists{parskip.sty}{%
    \usepackage{parskip}
  }{% else
    \setlength{\parindent}{0pt}
    \setlength{\parskip}{6pt plus 2pt minus 1pt}}
}{% if KOMA class
  \KOMAoptions{parskip=half}}
\makeatother
\usepackage{xcolor}
\usepackage[margin=1in]{geometry}
\usepackage{graphicx}
\makeatletter
\def\maxwidth{\ifdim\Gin@nat@width>\linewidth\linewidth\else\Gin@nat@width\fi}
\def\maxheight{\ifdim\Gin@nat@height>\textheight\textheight\else\Gin@nat@height\fi}
\makeatother
% Scale images if necessary, so that they will not overflow the page
% margins by default, and it is still possible to overwrite the defaults
% using explicit options in \includegraphics[width, height, ...]{}
\setkeys{Gin}{width=\maxwidth,height=\maxheight,keepaspectratio}
% Set default figure placement to htbp
\makeatletter
\def\fps@figure{htbp}
\makeatother
\setlength{\emergencystretch}{3em} % prevent overfull lines
\providecommand{\tightlist}{%
  \setlength{\itemsep}{0pt}\setlength{\parskip}{0pt}}
\setcounter{secnumdepth}{-\maxdimen} % remove section numbering
\ifLuaTeX
  \usepackage{selnolig}  % disable illegal ligatures
\fi
\usepackage{bookmark}
\IfFileExists{xurl.sty}{\usepackage{xurl}}{} % add URL line breaks if available
\urlstyle{same}
\hypersetup{
  pdftitle={Final Paper},
  pdfauthor={Owen Erker},
  hidelinks,
  pdfcreator={LaTeX via pandoc}}

\title{Final Paper}
\author{Owen Erker}
\date{2025-05-08}

\begin{document}
\maketitle

\section{Evaluating Trade-offs in Multi-Label Chest X-ray Classification
with
CNNs}\label{evaluating-trade-offs-in-multi-label-chest-x-ray-classification-with-cnns}

\textbf{Author:} Owen Erker\\
\textbf{Institution:} University of {[}Your Institution Here{]}\\
\textbf{Course:} BMI/CS 567 Medical Image Analysis\\
\textbf{Date:} April 11, 2025

\subsection{Abstract}\label{abstract}

Chest X-ray imaging is critical for diagnosing a wide range of thoracic
diseases in clinical settings. In this project, we use the ChestMNIST
dataset, derived from the NIH ChestXray14 collection, to evaluate
multi-label classification performance under varying image resolutions.
We implement a custom convolutional neural network (CNN) architecture
and assess its performance across 28×28, 64×64, 128×128, and 224×224
resolutions. Our findings show that low-resolution models (e.g., 28×28
and 64×64) maintain competitive diagnostic accuracy, achieving AUCs
around 0.74, with minimal degradation in performance relative to
high-resolution models.

\subsection{Introduction}\label{introduction}

Medical imaging datasets like ChestXray14 are massive and
computationally demanding to process. In clinical environments,
lower-resolution image processing could offer dramatic resource savings,
but the extent to which this sacrifices diagnostic accuracy is not well
understood. This project investigates how CNN models perform when
trained on chest X-ray images resized to different resolutions, using
the ChestMNIST dataset for standardized, reproducible evaluation.

To explore this question, we trained a custom CNN model across multiple
image sizes and analyzed how performance metrics such as AUC, F1 score,
and accuracy vary. The core hypothesis is that even at lower image
resolutions, meaningful diagnostic features are retained and can be
leveraged by deep learning models.

Our results suggest that significant performance can be maintained even
under aggressive image downsampling. This finding has implications for
designing lightweight, scalable AI systems in healthcare. The study also
surfaces areas for improvement, such as rare class detection and model
regularization strategies.

\subsection{Methods}\label{methods}

\subsubsection{Dataset Description}\label{dataset-description}

The dataset used is ChestMNIST, a preprocessed version of the NIH
ChestXray14 dataset, made available through the MedMNIST Python package.
It consists of 2D grayscale chest X-ray images labeled with 14 binary
disease indicators. Each label corresponds to the presence or absence of
a thoracic condition such as pneumonia, emphysema, or cardiomegaly.
Images are available in four resolutions: 28×28, 64×64, 128×128, and
224×224. Data splits are predefined for training, validation, and test
sets. No additional resampling or external preprocessing was applied.

\subsubsection{CNN Architecture}\label{cnn-architecture}

We implemented a modular convolutional neural network class called
\texttt{ChestMNIST\_CNN} in PyTorch. The model dynamically adjusts its
depth and complexity depending on the input resolution. For 28×28 and
64×64 images, it uses two convolutional blocks. For higher resolutions
(128×128, 224×224), additional layers are appended to increase channel
depth and receptive field size. Each convolutional block consists of
convolution, batch normalization, ReLU activation, max pooling, and
dropout. A final fully connected layer with sigmoid activation produces
14 outputs corresponding to the disease classes.

\subsubsection{Training Procedure}\label{training-procedure}

For each resolution, the model was trained for up to 100 epochs using
the Adam optimizer (learning rate = 1e-3). The loss function was
\texttt{BCEWithLogitsLoss}, enhanced with per-class positive weights to
correct for class imbalance. These weights were optionally clipped to a
maximum of 10 to prevent instability. A global classification threshold
of 0.3 was initially applied, but we also computed per-class optimal
thresholds using ROC curve analysis after training.

We logged training and validation loss, accuracy, AUC, macro and micro
F1 scores at each epoch. The best model (based on validation accuracy)
was saved, and per-class thresholds were stored for evaluation.

\subsubsection{Evaluation Metrics}\label{evaluation-metrics}

We used accuracy, AUC (Area Under the ROC Curve), macro-F1, and micro-F1
scores to evaluate model performance. Macro-F1 provides equal weight to
all classes, reflecting balanced performance, while micro-F1 aggregates
predictions across all labels. AUC summarizes probabilistic prediction
quality.

\subsection{Results}\label{results}

Our experiments show that models trained on lower-resolution images can
still perform competitively. At 28×28 resolution, validation accuracy
reached \textasciitilde70\%, AUC \textasciitilde0.74, and micro-F1
\textasciitilde0.22. Performance improved modestly with increasing
resolution, peaking around 128×128 and 224×224 with AUCs near 0.76 and
micro-F1 scores around 0.24. However, the gains were marginal relative
to the increased computational demands.

Training loss declined consistently across all resolutions, and
validation loss plateaued earlier for lower-resolution models. This
indicates fast convergence with limited overfitting. The use of
per-class optimal thresholds improved F1 scores and helped correct for
class imbalance.

Notably, macro-F1 scores remained modest (0.14--0.17), reflecting
challenges in detecting rarer pathologies. Table 1 (omitted here)
summarizes the best validation metrics at each resolution. The results
indicate that lightweight models with reduced image inputs can still
provide clinically meaningful predictions.

\subsection{Conclusions}\label{conclusions}

Our study demonstrates that convolutional neural networks trained on
aggressively downsampled chest X-ray images (28×28, 64×64) maintain
strong diagnostic performance, achieving AUCs around 0.74 and micro-F1
scores near 0.22. While models trained on higher resolutions showed
slightly better metrics, the marginal improvements must be weighed
against increased resource usage.

The best-performing method was the CNN trained at 128×128 resolution,
balancing accuracy (\textasciitilde72\%), AUC (\textasciitilde0.76), and
training stability. Its performance gain over 28×28 models was small,
suggesting that resolution alone is not the dominant factor. With more
time or computing power, we would explore deeper architectures, hybrid
models (e.g., CNN-transformers), and ensemble approaches.

\subsubsection{Limitations and
Disclosures}\label{limitations-and-disclosures}

This study is limited by its reliance on a single dataset and absence of
external validation. Models also struggled to detect rare diseases,
limiting macro-F1 scores. We did not quantify runtime or memory usage,
which would be important for deployment.

Generative AI was used to assist with code structure, text formatting,
and interpretation summaries. All modeling, training, and evaluation
code was implemented by the author.

\subsection{References}\label{references}

\begin{enumerate}
\def\labelenumi{\arabic{enumi}.}
\tightlist
\item
  Yang, J., Shi, R., Wei, D., et al.~``MedMNIST v2: A large-scale
  lightweight benchmark for 2D and 3D biomedical image classification,''
  \emph{Scientific Data}, 2023.
  \url{https://doi.org/10.1038/s41597-022-01721-8}\\
\item
  Yang, J., Shi, R., and Ni, B. ``MedMNIST Classification Decathlon,''
  in \emph{Proc. IEEE Int. Symp. Biomedical Imaging (ISBI)}, 2021.
\end{enumerate}

\end{document}
